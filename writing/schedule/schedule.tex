\documentclass[10pt]{article}

\usepackage[table]{xcolor}
\definecolor{Gray}{gray}{.7}

\begin{document}
\pagenumbering{gobble}
\noindent
{\Large\textbf{Phylogenetic Reconstruction Proposed Schedule}}\\
Evan Albright, Jack Hessel, Nao Hiranuma, Cody Wang
\\
There are several paradigms we hope to follow throughout our work on
comps. When we assign a particular algorithm to a group, this means
that the group implements, reviews, analyzes, and writes a section
for the final paper on that algorithm. Furthermore, the Wikidot/SVN
should be consistently updated in accordance with the models we have
developed/will develop when we create a class diagram (Week 5, Fall).
In the end, we should have very little work to do to ``put everything
together,'' because we will have a cohesive piece of software,
a fully updated paper with bibliographic information, and a wiki
full of content related to our project.\\


\noindent \textbf{Fall Term}\\\\
\noindent
\rowcolors{1}{Gray}{white}
\begin{tabular}{ | l | p{10cm} |}
\hline
Week 5 & Class Diagram (Simple UML or Hand Drawn) for Thursday, implemented
basic C++ data structures for Sunday.\\ 
\hline
Week 6 & Split up: 2 people on Neighbor Joining, 2 people on Data Generator\\
\hline
Week 7 & Split up: 2 people on Maximum Likelihood, 2 people finishing Neighbor
Joining/Data Generator and implementing Maximum Parsimony\\
\hline
Week 8 & Split up: 2 people on finishing Maximum Likelihood, 2 people on finishing
Maximum Parsimony\\
\hline
Week 9 & Split up: 2 people on Clustal-W, 2 People on MSA\\
\hline
Week 10 & Buffer Week/Code Parallelization Week (depending on where
we are)\\
\hline
\end{tabular}
\\\\
\noindent \textbf{Winter Break}\\
\begin{enumerate}
  \item
    Bug Fix for Maximum Likelihood and MSA.
  \item
    Get real-world data (genbank).
  \item
    Look for another algorithm other than Neural Network.
  \item
    Look at Parallelization stuff.
  \item
    Finish up paper writing for implemented algorithms.
  \item
    Update using RawSequence to using AlignedSequence.
  \item
    Get Options stuff working.
\end{enumerate}
Winter break can effectively be seen as a buffer time. If there are major
deviations from the schedule during fall term that cause us to not
accomplish what we would like to, this is the time to fix that. If there
aren't major deviations, people can optionally investigate individually
interesting topics that we might not have allocated proper time for
in the schedule (ie. parallelization, real world application, etc.)
\\\\
\noindent \textbf{Winter Term}\\\\
\noindent
\rowcolors{1}{Gray}{white}
\begin{tabular}{ | l | p{10cm} |}
\hline
Week 1-3 & Advanced Reconstruction Algorithms: 2 people on Monte-Carlo Markov Chain,
2 people on another advanced algorithm.\\ 
\hline
Week 4 & Buffer week: can be merged into weeks 1-3 or weeks 5-6 as needed.\\
\hline
Week 5-6 & Split up: 2 on code optimization (parallelization, etc) and 2 on applying implemented
algorithms to real world data.\\
\hline
Week 7 & Presentation preparation/Finish up coding/Debugging\\
\hline
Week 8 & Paper finalization\\
\hline
Week 9-10 & Code/paper finalization, more optimization/last minute work\\
\hline
\end{tabular}

\end{document}
